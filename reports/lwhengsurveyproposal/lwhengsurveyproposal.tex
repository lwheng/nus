\documentclass[12 pt]{article}
\usepackage{amssymb,amsmath,amstext,amsgen,amsbsy,amsopn,amsfonts,graphicx,theorem, anysize, multicol}
%\renewcommand{\baselinestretch}{1.0}
%\marginsize{0.5in}{0.5in}{0.5in}{0.5in}
%\marginsize{left}{right}{top}{bottom}
%\usepackage[lmargin=2cm,,rmargin=2cm,tmargin=2cm,bmargin=2cm]{geometry}
\usepackage[margin=2cm]{geometry}
% By default, \columnseprule is 0pt, which means they are invisible, to
% change:
%\setlength{\columnseprule}{0.5pt}

% If you feel the columns are too close together (default: 10pt), then
% redefine:
%\setlength{\columnsep}{20pt}

\begin{document}
\bibliographystyle{acm}
\title{CS2309 \\ Survey Paper Proposal}
\author{Heng Low Wee \\ U096901R}
\date{}
\maketitle

\indent \indent The topic that I have chosen is \textbf{Word Sense Disambiguation} (WSD). Most languages, for example English, have words or phrases that carry multiple meanings when used in different context. Looking at this example, \textit{I am interested in the interest rates of the bank}, one can tell that the word \textit{interest} can be related either to the state of wanting and curiosity or to the financial term related to debts. Similarly, \textit{bank} can mean the bank where we deposit our money, or a river bank. Of course, it is easy for a human to identify the appropriate sense for a certain context, but not so easy for a machine.
\\
\\
\indent \indent In order to make language translations more accurate and relevant, word-by-word translations must be able to correctly opt for the same contextual sense. My focus would be on some methods of Unsupervised WSD, comparing these methods and some of the advantages they have.
\\
\\
\indent \indent With that, my proposed list of papers is as referenced\nocite{unsupervised}\nocite{wikipedia}\nocite{noisy} below.
\bibliography{bibsource}
\end{document}













