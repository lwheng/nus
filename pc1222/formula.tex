\documentclass[10 pt]{article}
\usepackage{amssymb,amsmath,amstext,amsgen,amsbsy,amsopn,amsfonts,graphicx, overcite,theorem, anysize, multicol}
\usepackage[compact]{titlesec} \titlespacing{\section}{0pt}{*0}{*0} \titlespacing{\subsection}{0pt}{*0}{*0} \titlespacing{\subsubsection}{0pt}{*0}{*0}
\renewcommand{\baselinestretch}{1.0}
%\marginsize{0.5in}{0.5in}{0.5in}{0.5in}
%\marginsize{left}{right}{top}{bottom}
\usepackage[lmargin=0.1in,rmargin=0.1in,tmargin=0.1in,bmargin=0.1in]{geometry}
% By default, \columnseprule is 0pt, which means they are invisible, to
% change:
\setlength{\columnseprule}{0.1pt}

% If you feel the columns are too close together (default: 10pt), then
% redefine:
\setlength{\columnsep}{10pt}

\makeatletter
\newenvironment{tablehere}
  {\def\@captype{table}}
  {}

\newenvironment{figurehere}
  {\def\@captype{figure}}
  {}
\makeatother

\begin{document}
\begin{multicols}{2}
\subsection*{Lecture 1: Electromagnetic Interactions}
\subsubsection*{Electric Force (Coulombs Law)}
$F = \frac{1}{4 \pi \epsilon_{0}} \cdot \frac{|q_{1}||q_{2}|}{r^{2}}$, Newtons (N) \\
$\epsilon_{0} = 8.85 \times 10^{-12}~C^{2}/Nm^{2}$, Permittivity of free space.\\
$\frac{1}{4 \pi \epsilon_0} = 8.99 \times 10^9 Nm^2/C^2$, Coulomb's Constant
\subsubsection*{Fundamentals of Charges}
Fundament charge: $e = 1.60 \times 10^{-19}~C$ \\
$q = (N_{p} - N_{e})e$ \\ 
$q$ : charge, $N_{p}$ : No. of protons, $N_{e}$ : No. of electrons
\\
\line(1,0){290}
\subsection*{Lecture 2: Electric Force and Field}
\subsubsection*{Electric Field}
$E = \frac{1}{4 \pi \epsilon_{0}} \cdot \frac{|q|}{r^{2}}$ \\ 
S.I unit: Newtons/Coulombs (N/C) \\
*Direction of electric field: positive (outwards), negative (inwards)

\subsubsection*{Large charged plate}
$E = \frac{\sigma}{2\epsilon_{0}}, \sigma = \frac{Q}{A}.~$\\
$\sigma$ is Surface Charge Density \\
$Q$ is charge. \\
$A$ is area of plate. \\
*No. of field lines entering/leaving the charge is proportional to the amount of charge.
\\
\line(1,0){290}
\subsection*{Lecture 3: Electric Potential}
\subsubsection*{Work done}
$W_{a\rightarrow b} = - \Delta U = U_{a} - U_{b}$
\subsubsection*{Potential energy of charge $q_{0}$ in a uniform electric field}
 $U = q_{0}Ey$ \\
$y$: distance from negative plate.

\subsubsection*{Electric potential energy between 2 point charges}
$U = \frac{1}{4 \pi \epsilon_{0}} \cdot \frac{q_{1} \cdot q_{2}}{r}$ \\
\normalsize
*For shells, $r$ is distance between centers. \\
$U = \frac{1}{4 \pi \epsilon_{0}} \cdot \sum_{i<j} \frac{q_{i} \cdot q_{j}}{r_{ij}}$ \\
$U = q \Delta V = q(V_{f} - V_{i})$

\subsubsection*{Electric Potential}
$V = \frac{1}{4 \pi \epsilon_{0}} \cdot \frac{q}{r}$ \\
\normalsize
*Electric \textbf{force} on a charge is always in the direction of \textbf{lower} electric \textbf{potential energy}. \\
*Electric \textbf{field} is always in the direction of \textbf{lower potential}. \\
$V = \frac{1}{4 \pi \epsilon_{0}} \cdot \sum_{i} \frac{q_{i}}{r_{i}}$ (scalar sum)

\subsubsection*{Total energy conservation}
$K_{i} + qV_{i} = K_{f} + qV_{f}$ \\
\normalsize
$K$: kinetic energy \\
$qV$: potential energy \\
$V$: potential

\subsubsection*{Kinetic Energy}
$E_{k} = \frac{1}{2}mv^{2}$ \\
\normalsize
*Mass of proton: $1.67 \times 10^{-27}~kg$ \\
*Mass of electron: $9.11 \times 10^{-31}~kg$

\subsubsection*{Electric Field vs Potential}
$E = \frac{\Delta V}{d}$ \\
\normalsize
$d$: distance between 2 equipotential surfaces
\\
\line(1,0){290}
\subsection*{Lecture 4: Capacitance}
*Any 2 \underline{conductors} (regardless of shape) separated by an insulator (or vacuum) form a Capacitor. \\
*The 2 conductors have charges with \underline{equal magnitude} but \underline{opposite signs}.
\subsubsection*{Capacitance}
$C = \frac{Q}{\Delta V}$, Coulomb per Volt (C/V), Farad (F). \\
$C = \frac{\epsilon_{0}A}{d}$, $A$: Area, $d$: distance between

\subsubsection*{Energy Storage in Capacitor}
$U = \frac{1}{2}QV = \frac{Q^{2}}{2C} = \frac{1}{2}CV^{2}$\normalsize, Joules (J)

\subsubsection*{Energy Density in Electric Field}
$u = \frac{1}{2}\epsilon_{0}E^{2}$, $(J/m^{3})$ \\
$u$: amount of \underline{energy per unit volume}

\subsubsection*{Capacitors with Dielectrics}
$C = \kappa C_{0}, \kappa > 1$ \\
$\kappa$: dielectric constant \\
\large Inserting dielectric \underline{without} battery \\
$\Rightarrow V = \frac{Q_{0}}{C} = \frac{Q_{0}}{\kappa C_{0}} = \frac{V_{0}}{\kappa}$ \\
$\Rightarrow E = \frac{V}{d} = \frac{V_{0}}{\kappa d} = \frac{E_{0}}{\kappa}$ \\
$\Rightarrow U = \frac{U_{0}}{\kappa} < U_{0}$ \\
\large Inserting dielectric \underline{with} battery \\
$\Rightarrow Q = CV_{0} = (\kappa C_{0})V_{0} = \kappa Q_{0}$ \\
$\Rightarrow U = \kappa U_{0} > U_{0}$ \\
*Dielectric strength of Dry Air = $3 \times 10^{6}~V/m$
\subsubsection*{Find Maximum Charge on Capacitor}
*Given dielectric strength of medium: \\
$Q_{max} = \kappa \times \frac{\epsilon_{0} A}{d} \times d$
\subsubsection*{Capacitor Network}
In parallel: $C = C_{1} + C_{2}$ \\
In series: $\frac{1}{C} = \frac{1}{C_{1}} + \frac{1}{C_{2}}$
\\
\line(1,0){290}
\subsection*{Lecture 5: Current and Resistance}
\subsubsection*{Ohm's Law}
$I = \frac{\Delta V}{R}$ \\
\large In series: \\
$V = V_{1} + \ldots + V_{n}$ \\
$I = I_{1} = \ldots = I_{n}$ \\
$R = R_{1} + \ldots + R_{n}$ \\
\large In parallel: \\
$V_{1} = V_{2} = \ldots = V_{n}$ \\
$I = I_{1} + \ldots + I_{n}$ \\
$\frac{1}{R} = \frac{1}{R_{1}} + \ldots + \frac{1}{R_{n}}$

\subsubsection*{Currents and Charges}
$I = \frac{\Delta Q}{\Delta t} = |q|nAv_{d}$ \\
\normalsize
$n$: no. of charge carriers per unit volume \\
$A$: cross section area \\
$v_{d}$: drift speed of current

\subsubsection*{Resistance}
$R = \frac{\rho L}{A}$ , (Ohms, $\Omega$) \\
$\rho$: resistivity, (Ohms meter, $\Omega \cdot m$) \\
$L$: length \\
$A$: cross section area

\subsubsection*{Resistance and Temperature}
$\rho = \rho_{0} [ 1 + \alpha (T - T_{0}) ]$ \\
$R = R_{0} [ 1 + \alpha (T - T_{0}) ]$ \\
\normalsize
$\rho_{0}$: resistivity at $T_{0}$ \\
$T_{0}$: usually $20^{\circ} C$ \\
$\alpha$: temp. coefficient of resistivity $(^{\circ}C)^{-1}$

\subsubsection*{Power}
$P = \frac{\Delta Q \Delta V}{\Delta t} = \Delta V I = I^{2} R = \frac{\Delta V^{2}}{R}$ , (Watts, W).
\\
\line(1,0){290}
\subsection*{Lecture 6: DC Circuits}
$I = \frac{\varepsilon}{R + r}$ \\
\normalsize
$\varepsilon$: EMF \\
$R$: external resistor \\
$r$: internal resistor (in battery)

\subsubsection*{Kirchhoff's Rules}
*$\sum I_{i}$ into junction = $\sum I_{i}$ out of junction \\
*Sum of voltage drops around a loop is zero, $\varepsilon = Ir + IR$ \\
*Intuition: Applying Kirchhoff is to form equations of $emf = 0$. \\
*Recall: Can use Row Operations from Linear Algebra to reduce to Reduced Row Echelon form.

\subsubsection*{RC Circuits}
\large RC Circuit: Charging \\
$\varepsilon - I(t)R - \frac{q(t)}{c} = 0$ \\
Just after switch closed: \\
$q(t) = q_{0} = 0 \Rightarrow \varepsilon - I_{0}R = 0 \Rightarrow I_{0} = \frac{\varepsilon}{R}$ \\
Long after switch closed: \\
$I(t) = I_{\infty} = 0 \Rightarrow \varepsilon - \frac{q_{\infty}}{C} = 0 \Rightarrow = q_{\infty} = C\varepsilon$ \\
Intermediate: \\
$q(t) = q_{\infty}(1-e^{-t/RC}),~I(t) = I_{0}e^{-t/RC}$ \\
\large RC Circuit: Discharging \\
$\frac{q(t)}{C} + I(t)R = 0$ \\
Just after switch closed: \\
$q(t) = q_{0} \Rightarrow \frac{q_{0}}{C} + I_{0}R = 0 \Rightarrow I_{0} = - \frac{q_{0}}{RC}$ \\
Long after switch closed: \\
$I(t) = I_{\infty} = 0 \Rightarrow \frac{q_{\infty}}{C} = 0 \Rightarrow q_{\infty} = 0$ \\
Intermediate: \\
$q(t) = q_{0}e^{-t/RC}, ~ I(t) = I_{0}e^{-t/RC}$
\\
\line(1,0){290}
\subsection*{Lecture 7: Magnetism}
\subsubsection*{Magnetic Flux}
$\Phi = BA \cos{\theta}$ , SI unit: $Tm^{2} \equiv Wb$ \\
$B$: magnetic field strength (Teslas, T) \\
$A$: area of plane \\
$\theta$: angle between $B$ and normal to plane

\subsubsection*{Magnetic Force on ONE moving charge}
$F = |q|vB\sin{\theta}$ , Newtons \\
$\theta$: angle between $v$ and $B$. \\
(recall: $E=\frac{F}{q} \Rightarrow E = vB$, when $\theta=90^\circ$)

\subsubsection*{Right-hand Rule}
Thumb: $F$, Index: $v$, Middle: $B$ \\
*If \underline{negative} charge, $F$ is in \underline{opposite} direction to thumb

\subsubsection*{Radius of circular orbit}
$R = \frac{mv}{B|q|}$ \normalsize
\\
\line(1,0){290}
\subsection*{Lecture 8: Currents and Magnetism}
\subsubsection*{Magnetic Force on MANY moving charges}
$F = ILB\sin{\theta}$ \\
$L$: length of wire \\
$\theta$: angle between $I$ and $B$
\subsubsection*{Gravity Force}
$F = mg$ \normalsize
\subsubsection*{Magnetic Field of a Long Straight Wire}
$B = \frac{\mu_{0} I}{2 \pi r}$ \\
$\mu_{0}$: permeability of vacuum, $\mu_{0} = 4\pi \times 10^{-7}~Tm/A$ \\
$r$: distance from wire \\
*Right-hand rule: Thumb: direction of $I$, fingers: direction of $B$.
\subsubsection*{Force between parallel wires with length L}
*Consider $I$ directed to the right, wire $a$ is $d$ meters from $b$. \\
Magnetic force on wire $a$ and $b$: \\
$F_{a} = I_{a}LB_{b} = \frac{\mu_{0}I_{a}I_{b}L}{2\pi d}$, (downward) \\
$F_{b} = I_{b}LB_{a} = \frac{\mu_{0}I_{a}I_{b}L}{2\pi d}$, (upward)

\subsubsection*{Bar magnet vs Current Loop}
*If wrap fingers around wire in loop, Thumb: $I$, fingers: $B$ \\
*If put fingers on loop, Thumb:$B$, fingers: $I$, like inside of a Bar Magnet \\
*The side of the current loop from which $B$ emerges is North pole.

\subsubsection*{Magnetic Field at CENTER of current loop}
$B = \frac{\mu_{0}NI}{2R}$ \\
$N$: number of loops

\subsubsection*{Interaction between Current Loops}
*Same $I$ direction: attract \\
*Different $I$ direction: repel

\subsubsection*{Magnetic Field INSIDE ideal solenoid}
$B = \mu_{0}nI$ \\
$n$: no. of turns per unit length \\
*Solenoid becomes like a Bar Magnet
\subsubsection*{Cosine Rule}
$c^2 = a^2 + b^2 - 2ab \cos{C}$, $C$ is angle between $a$ and $b$ \\
\line(1,0){290}
\subsection*{Lecture 9: Electromagnetic Induction}
$\Delta \Phi \rightarrow \varepsilon_{induced} \rightarrow I_{induced} \rightarrow B_{induced} \rightarrow \Phi_{induced}$
\subsubsection*{Len's Law} 
The direction of $I_{induced}$ is such that the $B_{induced}$ \textbf{opposes} the \textbf{change} in the flux that induces the current.
\subsubsection*{Faraday's Law of Induction}
$|\varepsilon| = N \frac{|\Delta \Phi|}{\Delta t} = N \frac{|\Phi_{f} - \Phi_{i}|}{\Delta t}$, $N$ is no. of loops
\subsubsection*{Motional Electromotive Force}
$\Delta V = \varepsilon = vBL$, (recall $E=vB, E=\frac{\varepsilon}{L}$)\\
To have motional emf, conductor must \textbf{cut} through magnetic field
\subsubsection*{Induced Current}
$I = \frac{\varepsilon}{R} = \frac{vBL}{R}$
\subsubsection*{Magnetic Force (magnitude) on bar}
$F =ILB = \frac{vB^{2}L^{2}}{R}$
\\
\line(1,0){290}
\subsection*{Lecture 10: Electromagnetic Waves (EM Waves)}
\subsubsection*{Speed of EM Waves}
$c = \frac{1}{\sqrt{\varepsilon_{0}\mu_{0}}}$, $E=cB$ \\
Transverse wave: $\vec{E}$ and $\vec{B}$ fields are \textbf{perpendicular}. 
\subsubsection*{Energy in EM Waves}
$u_{E} = \frac{1}{2}\epsilon_{0}E^{2}, u_{B} = \frac{1}{2} \frac{B^{2}}{\mu_{0}}, u_{E} = u_{B}$ \\
So total, $u_{EM} = u_{E} + u_{B} = \epsilon_{0}E^{2} = \frac{B^{2}}{\mu_{0}}$
\subsubsection*{Intensity of EM Waves}
$I = \frac{U}{A \Delta t} = \frac{P}{A} = u_{EM}c = c\varepsilon_{0}E^{2} = \frac{c}{\mu_{0}B^{2}}$, (Watts/$m^{2}$ or $W/m^{2}$) \\ Here $I$ is Intensity, not Current
\subsubsection*{EM Spectrum}
$c = f\lambda$, $f$: frequency, $\lambda$: wavelength
\subsubsection*{Doppler Equation for EM Waves}
$f_{o} = f_{s} (1 \pm \frac{u}{c})$, only valid for $u \ll c$ \\
$f_{o}$ : observed frequency \\
$f_{s}$ : frequency emitted by the source \\
$u$ : relative speed between source and observer \\
$u = |v_{1}+v_{2}| \Rightarrow$ moving in opposite directions \\
$u = |v_{1}-v_{2}| \Rightarrow$ moving in same directions \\
*Important intuition: As objects come closer, frequency increase
\subsubsection*{Polarization of EM Waves}
*Direction of polarization of an EM wave is defined to be the \textbf{direction of $\vec{E}$ field}. A wave propagates with FIXED polarization $\rightarrow$ linearly polarized.
\subsubsection*{Unpolarized Light on Linear Polarizer}
$I_{transmitted} = \frac{1}{2} I_{incident}$, here $I$ is Intensity
\subsubsection*{Linearly Polarized Light on Linear Polarizer, Malus' Law}
$E_{transmitted} = E_{incident}\cos{\theta} \Rightarrow I_{transmitted} = I_{incident} \cos^{2}{\theta}$ \\
$\theta$: angle between the incoming light's polarization and the transmission axis (TA)
\subsubsection*{Optical Activity}
The ability of a substance to rotate the polarization direction of linearly polarization light. This ability depends on the molecular structure of the substance.\\
\line(1,0){290}
\subsection*{Lecture 11: Reflection Of Light}
\subsubsection*{Law of Reflection}
$\theta_{i} = \theta_{r}$ \\
$\theta_{i}$: Incident angle \\
$\theta_{r}$: Reflected angle \\
Specular vs Diffuse reflection: one is smooth surface, one is rough surface. Diffuse makes dry road easy to see at night. \\
*For plane mirrors, basically it's all about similar triangles
\subsubsection*{Spherical Mirrors}
$C = $ center of curvature.
\subsubsection*{Focal Point for Concave/Convex Mirror}
Concave: $f = \frac{R}{2}$, Convex: $f=-\frac{R}{2}$  \\
Concave converges, Convex diverges \\
$f$: focal length, distance from Focal Point, $F$, to center of mirror \\
$R$: distance from $C$ to center of mirror \\
*Flat/Plane mirror has $f = \infty, \because R\rightarrow \infty$
\subsubsection*{Ray tracing for Concave/Convex mirror \\(Find real/virtual image)}
Ray\#1: Parallel to principal axis, reflects through $F$. \\
Ray\#2: Through $C$, center of curvature. \\
Ray\#3: Through $F$, reflects parallel to principal axis. \\
*Intersection point is where the real/virtual image lies.
\subsubsection*{Spherical Mirror Equations}
$\frac{1}{d_{o}} + \frac{1}{d_{i}} = \frac{1}{f}$, applies to concave/convex \\
$d_{o}$: object distance
$d_{i}$: image distance
$f$: focal length
\subsubsection*{Magnification}
$m \equiv \frac{h_{i}}{h_{o}} = - \frac{d_{i}}{d_{o}}$ \\
$h_{o}$: object height \\
$h_{i}$: image height, $(-ve) \rightarrow$ inverted image \\
$m$ has same sign as $h_{i}$, $|m|<1 \rightarrow$ reduced, $|m|>1 \rightarrow$ enlarged. \\
\line(1,0){290}
\subsection*{Lecture 12: Refraction Of Light}
\subsubsection*{Index of Refraction}
$n \equiv \frac{c}{v}$, $n>1, \because c>v$ \\
$c$: speed of light in vacuum \\
$v$: speed of light in medium
\subsubsection*{Frequency between media}
$v=f\lambda \Rightarrow \frac{\lambda_{1}}{\lambda_{2}} = \frac{v_{1}}{v_{2}} = \frac{n_{2}}{n_{1}}$
\subsubsection*{Snell's Law of Refraction}
$n_{1}\sin{\theta_{1}} = n_{2}\sin{\theta_{2}}$ \\
$\theta$: angle between light and normal of incidence
\subsubsection*{Apparent Depth}
$\frac{d'}{d} = \frac{n_{1}}{n_{2}}$ \\
$d'$: apparent depth \\
$d$: actual depth \\
$n_{1}$: refraction index of medium 1 (before refraction) \\
$n_{2}$: refraction index of medium 2 (after refraction)
\subsubsection*{Total Internal Reflection}
Critical angle: $n_1 \sin{\theta_c} = n_2 \sin{90^{\circ}} \Rightarrow \sin{\theta_{c}} = \frac{n_{2}}{n_{1}}$
\subsubsection*{Polarization by Reflection, Brewster's Angle}
$\tan{\theta_{B}} = \frac{n_{2}}{n_{1}} \Rightarrow \theta_{B}:$ Brewster's Angle \\
*When unpolarized light is reflected from a surface, it \textit{can} get polarized at $0^{\circ} < \theta_{i}< 90^{\circ}$. At $0^{\circ}/90^{\circ}$, light remains unpolarized. \\
*Reflected light is \textbf{totally} polarized \textbf{parallel} to the surface when the reflected and refracted rays are at right angles. \\
*\textit{Dispersion} is the dependence of the index of refraction of a transparent medium on the wavelength of light.
\\
\line(1,0){290}
\subsection*{Lecture 13: Optical Instruments}
\subsubsection*{Thin Lenses}
Principal Rays for Converging/Diverging lens \\
Ray\#1: Parallel to principal axis, passes through $F$. \\
Ray\#2: Through center of lens. \\
Ray\#3: Through $F$, emerges parallel to principal axis.
\subsubsection*{Thin Lens Equation}
$\frac{1}{d_o} + \frac{1}{d_i} = \frac{1}{f}$ \\
$d_o$: object distance \\
Positive: real object (in front of lens) \\
Negative: virtual object (behind lens) \\
$d_i$: image distance \\
Positive: real image (behind lens) \\
Negative: virtual image (in front of lens) \\
$f$: focal length \\
Positive: convex (converging) lens \\
Negative: concave (diverging) lens
\subsubsection*{Magnification equation}
$m \equiv \frac{h_{i}}{h_{o}} = - \frac{d_{i}}{d_{o}}$ \\
$h_{o}$: object height \\
$h_{i}$: image height, $(-ve) \rightarrow$ inverted image \\
$m$ has same sign as $h_{i}$, $|m|<1 \rightarrow$ reduced, $|m|>1 \rightarrow$ enlarged.
\subsubsection*{Combination of Lenses}
$m_{total} = m_1 * m_2$ \\
$m_i$: magnification by $i$-th lens
\subsubsection*{Human Eyes}
Focal length of normal human eye $\approx$ 2.5cm \\
Near point of normal human eye $\approx$ 25cm \\
Far point of normal human eye : $\infty$ \\
When you are \textbf{near-sighted}: \\
1. Far point is too close \\
2. Need concave (diverging) lens to create an virtual image at that abnormal far point (which is $< \infty$), which acts as new object for your eye \\
When you are \textbf{far-sighted}: \\
1. Near point is too far \\
2. Need convex (converging) lens to create an virtual image at that abnormal near point (which is $> 25$cm), which acts as new object for your eye
\subsubsection*{Refractive Power of Lens}
$P = \frac{1}{f}$, Unit: Diopter, $(m^{-1})$ \\
$f$: focal length of lens
\subsubsection*{Magnifying Glass}
\textbf{Angular size}, \\
$\tan{\theta} = \frac{h_o}{d_o} \Rightarrow \theta \approx \frac{h_o}{d_o} \textrm{ or } \theta \approx \frac{h_o}{N}$ \\
$h_o$: object height \\
$N$: near point distance \\
As object is moved closer to eye, $\theta$ decreases. \\
Also, \\
$\theta \approx \frac{h_i}{d_i} \approx \frac{h_o}{d_o}$, (similar triangles) \\
\textbf{Angular Magnification}, \\
$M = \frac{\theta}{\theta_o}$ \\
$\theta_o$: angular size of object, (without magnifying glass) \\
$\theta$: angular size of image, (with magnifying glass) \\
$M = N(\frac{1}{f} - \frac{1}{d_i})$ \\
$\Rightarrow M = N(\frac{1}{f} - \frac{1}{-\infty})$ for virtual image at $\infty$ \\
$\Rightarrow M = N(\frac{1}{f} - \frac{1}{-N})$ for virtual image at $N$ \\
\line(1,0){290}

\subsection*{Lecture 14: Interference of Light}
\textbf{Principle of superposition} \\
Constructive superposition: $\lambda_2 - \lambda_1 = m\lambda, ~m=0,1,2,\ldots$ \\
Destructive superposition: $\lambda_2 - \lambda_1 = (m+\frac{1}{2})\lambda, ~m=0,1,2,\ldots$
\subsubsection*{Double Slit Interference}
Optical Path Difference: $\delta = d\sin{\theta}$
\subsubsection*{Constructive Interference (Bright Fringes)}
$\delta = d\sin{\theta} = m\lambda, ~m=0,1,2,\ldots$
\subsubsection*{Destructive Interference (Dark Fringes)}
$\delta = d\sin{\theta} = (m+\frac{1}{2})\lambda, ~m=0,1,2,\ldots$ \\
$d$: slit-spacing \\
$\theta$: angle between ray and norm \\
\textbf{*Center Bright} is $m=0$, \textbf{1st Bright} usually means $m=1$ \\
\textbf{*First Dark} from Center Bright is $m=0$ \\
\textbf{Small Angle Approx. To Find Bright/Dark Fringes} \\
Bright: $\theta \approx \frac{m\lambda}{d}, ~y \approx \frac{m\lambda L}{d}$ $\vline$ Dark: $\theta \approx (m+\frac{1}{2})\frac{\lambda}{d}, ~y \approx (m+\frac{1}{2})\frac{\lambda L}{d}$ \\
$L$: distance between slit and screen
\subsubsection*{Thin Film Interference}
\begin{tablehere}
%\centering
%\footnotesize
\begin{tabular} {| c | c | c |}
\hline
\textbf{Reflection} & \textbf{Distance} & \textbf{Optical path diff}\\
\hline
Ray 1 $0(n_0 > n_1)$ 					&	0			&	$\delta_1 = 0+0 (n_0 > n_1)$ \\
Ray 1 $\frac{\lambda}{2}(n_0 < n_1)$ 	&	0			&	$\delta_1 = \frac{\lambda}{2}+0 (n_0 < n_1)$ \\
Ray 2 $0(n_1 > n_2)$ 					&	$2n_1 t$	&	$\delta_1 = 0+2n_1 t (n_1 > n_2)$ \\
Ray 2 $\frac{\lambda}{2}(n_1 < n_2)$ 	&	$2n_1 t$	&	$\delta_1 = \frac{\lambda}{2}+2n_1 t (n_1 < n_2)$ \\
\hline
\end{tabular}
\end{tablehere} \\
$\delta = | \delta_1 - \delta_2| $: optical path difference between Ray 1 and 2 \\
$\delta_1$: optical path difference between Ray 1 and incident ray \\
$\delta_2$: optical path difference between Ray 2 and incident ray \\
Constructive: $\delta = m\lambda$, Destructive: $\delta = (m+\frac{1}{2})\lambda$, $m=0,1,\ldots$
\line(1,0){290}

\subsection*{Lecture 15: Diffraction and Grating}
\subsubsection*{Single Slit Diffraction - Locating Minima (Dark Fringe)}
$\sin{\theta} = \frac{m\lambda}{W}, ~y = \frac{m\lambda L}{W}$, $m=1,2,3,\ldots$ \\
*\textbf{First Dark} is $m=1$ (Notice the difference between Double Slit)
\subsubsection*{Diffraction from Circular Aperture}
$\sin{\theta} = 1.22\frac{\lambda}{D}$ \\
$\rightarrow$\textbf{First minimum} of a circular diffraction pattern \\
$D$: diameter of circular aperture
\subsubsection*{Resolving Power (Rayleigh's Criterion)}
$\theta_{min} = 1.22 \frac{\lambda}{D}$, \textbf{NOTE}: $\theta_{min}$ will be in Radians \\
$\theta_{min}$: angle between the 2 objects / light sources \\
$\lambda$: the wavelength \textit{in} the region between the aperture and screen \\
Two point objects are \textit{just resolved} when the 1st dark fringe in the diffraction pattern of one falls directly on the central bright fringe in the diffraction pattern of the other. So, to see 2 objects distinctively, we need: \\
$\theta_{objects} \geq \theta_{min}$, and $\theta_{objects} \approx \frac{d}{y}$ \\
$d$: distance between objects \\
$y$: distance from objects to the aperture
\subsubsection*{Convert Radians to Degree}
$\theta^{\circ}=\frac{\theta \text{ rad} \times 180^{\circ}}{\pi}$
\subsubsection*{Diffraction Grating}
$d \sin{\theta} = m\lambda$ \\
*Condition for constructive interference \\
$\theta = \sin^{-1}{\frac{m\lambda}{d}}, ~y = L \tan{\theta}$ \\
*\textbf{Center Bright} fringe is $m=0$ \\
*As no. of slits per unit length of grating $\uparrow$, fringes get narrower and brighter \\
$m$: $m=0,1,2,\ldots$ \\
$y$: distance from center fringe to $i$-th fringe \\
$L$: distance from grating to screen \\
\line(1,0){290}

\subsection*{Lecture 16: Photoelectric Effect}
\subsubsection*{Photoelectric Equation}
$K_{max} = hf - W_0 \Rightarrow K_{max} = h\frac{c}{\lambda} - W_0$ (units: $eV$) \\
$eV$: electron volt, 1 electron charge times 1 V \\
$K_{max}$: max kinetic energy of released electrons \\
$h$: \textbf{Planck's constant} = $6.63 \times 10^{-34} J s$ (Joules second)\\
$f$: frequency of light source \\
$W_0$: work function of the metal surface (minimum energy needed to free electron) \\
$f_0 = \frac{W_0}{h}, \lambda_0 = \frac{hc}{W_0}$ \\
$f_0, \lambda_0$: cutoff frequency/wavelength (if $\lambda_0 \uparrow$ then $W_0\downarrow$, no photoelectric) \\
*Remember: $K_{max}$ depend \textbf{only} on frequency and work function, not intensity \\
*Photon energy $>$ $W_0$ $\rightarrow$ electron freed \\
*Photon energy $<$ $W_0$ $\rightarrow$ electron not freed regardless of intensity
\subsubsection*{Stopping Potential}
$eV_0 = K_{max}, V_0 = \frac{h}{e}(f - f_0)$
\subsubsection*{de Broglie's Matter Wave}
$\lambda = \frac{h}{p} = \frac{h}{mv} = \frac{h}{\sqrt{2meV}}$ \\
$p$: momentum \\
$m$: mass \\
$v$: velocity \\
$e$: electron charge \\
$V$: potential \\
*Note: $eV$ together may mean, for e.g. 10.2 eV. Need to multiply by electron charge
\subsubsection*{Finding No. of photons per second}
$\frac{\#photons}{second} = \frac{energy/second}{energy/photon} = \frac{power}{energy/photon} = \frac{power}{hf}$ \\
\line(1,0){290}

\subsection*{Lecture 17: Atomic Physics}
\textbf{Line Spectra} \\
Absorption Spectrum: \\
Bright background with dark lines (Photons were absorbed) \\
Emission Spectrum: \\
Dark background with bright lines (Photons were emitted)
\subsubsection*{Balmer Series (Hydrogen)}
$\frac{1}{\lambda} = R_H (\frac{1}{2^2} - \frac{1}{n^2}), ~n=3,4,5,\ldots$ \\
*When $n=3$, $\lambda$ is longest.
\subsubsection*{Lyman Series (Ultraviolet)}
$\frac{1}{\lambda} = R_H (\frac{1}{1^2} - \frac{1}{n^2}), ~n=2,3,4,\ldots$
\subsubsection*{Paschen Series (Infrared)}
$\frac{1}{\lambda} = R_H (\frac{1}{3^2} - \frac{1}{n^2}), ~n=4,5,6,\ldots$
\subsubsection*{Brackett Series (Infrared)}
$\frac{1}{\lambda} = R_H (\frac{1}{4^2} - \frac{1}{n^2}), ~n=5,6,7,\ldots$
\subsubsection*{Pfund Series (Infrared)}
$\frac{1}{\lambda} = R_H (\frac{1}{5^2} - \frac{1}{n^2}), ~n=6,7,8,\ldots$
\subsubsection*{In general... (for the spectral lines in Hydrogen atom)}
$\frac{1}{\lambda} = R_H (\frac{1}{n_f^2} - \frac{1}{n_i^2}), ~n\in \mathbb{Z}^+,n_i > n_f$ \\
$R_H$: Rydberg Constant, $1.097 \times 10^7 m^{-1}$
\subsubsection*{Bohr's Model of Atom}
Intuition: \\
Electron jump from level $n=i$ to $n=i-1 \rightarrow$ emits photons \\
Electron jump from level $n=i$ to $n=i+1 \rightarrow$ absorbs photons
\subsubsection*{Bohr's Angular Momentum Quantization}
$2 \pi r = n\lambda ~\&~ \lambda = \frac{h}{p} \Rightarrow mrv = \frac{nh}{2 \pi}$
\subsubsection*{Planck-Einstein / Planck's relation}
$E = hf = h \frac{c}{\lambda} \Rightarrow \lambda = \frac{hc}{E}$ \\
$h$: Planck's constant, $6.63 \times 10^{-34} J s$ \\
$f$: frequency \\
$E$: energy, so can be in $eV$ or $J$. (Recall equation in Lect 16)
\subsubsection*{Energy Levels of Hydrogen Atom}
$E_n = - \frac{1}{2} \frac{1}{4 \pi \epsilon_0} \frac{e^2}{n^2 a_0} = - \frac{13.6 \textrm{ eV}}{n^2}, ~(a_0 = \epsilon_0 \frac{h^2}{\pi m e^2}, ~r_n = n^2 a_0)$ \\
$a_0$: smallest orbit radius, \textit{Bohr radius} = 0.0529nm \\
$r_n$: general expression for radius of any orbit \\
*Lowest energy level or \textit{ground state} $\rightarrow n=1$ \\
*\textit{First excited state} $\rightarrow n=2$, and so on... \\
*Highest level, electron removed $\rightarrow n=\infty, E=0$
\subsubsection*{Some Intuition about Spectral Lines}
When there is a transition from (e.g.) $n=3$ to $n=2$, then there is a wavelength for that photon(s). So for e.g. electron excited to $n=3$. $(3\rightarrow 2), (2\rightarrow 1) ~\& ~ (3\rightarrow 1)$ are spectral lines.
\subsubsection*{Energy Levels and Spectrum}
$\frac{1}{\lambda} = R_H |\frac{1}{n_f^2} - \frac{1}{n_i^2}|$ \\
*Longest $\lambda$: e.g. $\lambda_{32} = \frac{hc}{E_3 - E_2}$ (Balmer series) \\
*Shortest $\lambda$: e.g. $\lambda_{\infty 2}$ (Balmer series)
\subsubsection*{In General, when there's $Z$/Protons involved..}
$E_n = - \frac{1}{4 \pi \epsilon_0} \frac{Z^2 e^2}{2n^2 a_0} = - \frac{Z^2 (13.6 \textrm{ eV})}{n^2}, (r_n = \frac{n^2 a_0}{Z})$ \\
$Z$: number of protons in nucleus \\
$\Rightarrow \frac{1}{\lambda} = Z^2 R_H |\frac{1}{n_f^2} - \frac{1}{n_i^2}|$ \\
\line(1,0){290}
\end{multicols}
Converging lenses
\begin{tabular}{|l|l | l| l |}
\hline
	$d_o<f$		& 	Up. 	& 	Mag. & Virt.\\
	$d_o=f$		&	$-$ 	&	$-$  &	$-$	\\
	$d_o=2f$	&	Inv.	&	Same & Real	\\
	$d_o>f$		&	Inv.	&	Dim. & Real	\\
\hline
\end{tabular}
\end{document}
